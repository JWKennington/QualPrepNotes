\newpage
%%%%%%%%%%%%%%%%%%%%%%%%%%%%%%%%%%%%%%%%%
%        ELECTRODYNAMICS SECTION        %
%%%%%%%%%%%%%%%%%%%%%%%%%%%%%%%%%%%%%%%%%
\chapter{Electricity \& Magnetism}
\label{sec:electro}
The electricity and magnetism subject matter in the qualifying exam is primarily at the level of \cite{griffithsIntroductionElectrodynamics2018}, a common, upper-level undergraduate electricity and magnetism text.


\section{Electrostatics}
We begin with the study of static electric fields. The electric field is one of two principle elements in electrodynamics, 
% Electric Field
\subsection{Electric Field}
% 	Superposition
The \textit{Principle of Superposition} in electrostatics states that the forces between two charges do not depend on any other charges. Consequently, multiple such forces and fields may be \textit{superimposed}, or added together, to construct the net force and field.
% 	Coulomb Law
Now that we know we can add forces between pairs of charges together, we must describe the force between two electric charges, \textit{Coulomb's law}:
\eqn{\vec{F} = k\frac{qQ}{r^2}\vec{\hat{r}} 
\quad\quad \text{where} \quad\quad k = \frac{1}{4\pi \epsilon_0} 
\quad\quad \text{and} \quad\quad \vec{r}=\vec{r}_Q - \vec{r}_q} 
In the above, $\epsilon$ is the \textit{permittivity of free space}. Note that for like sign charges, $\vec{F}$ is repulsive, and for dislike sign charges, $\vec{F}$ is attractive.
% 	Electric Field
The above equations also make use of the \textit{test charge} $Q$, but it is also possible to discuss the vector field $\vec{E}$ created by the source charges $q$ by the following:
\eqn{\vec{F} = Q\vec{E} 
\quad\quad \text{where} \quad\quad 
\vec{E}(\vec{r}) = k \sum_i \frac{q_i}{r_i^2}\vec{\hat{r}}_i = k \int \frac{1}{r^2}\vec{\hat{r}}\ dq,
\quad \vec{r}_i = \vec{r} - \vec{r}_{q_i}}
% 	Continuous Charge Distribution
For continuous charge distributions, the above can be rewritten in terms of various charge \textit{densities}. In the linear case, a line charge density $\lambda(\vec{r})$ is used. Similarly, for surface charges, $\sigma(\vec{r})$ is used, and for volume charges $\rho(\vec{r})$ is used.
\eqn{\begin{aligned}
	\vec{E}(\vec{r}) &= k \int \frac{\lambda\wrap{\vec{r}'}}{r^2}\vec{\hat{r}}\ dl' \\
	\vec{E}(\vec{r}) &= k \int \frac{\sigma\wrap{\vec{r}'}}{r^2}\vec{\hat{r}}\ dS' \\
	\vec{E}(\vec{r}) &= k \int \frac{\rho\wrap{\vec{r}'}}{r^2}\vec{\hat{r}}\ dV' \\
\end{aligned}}
% Divergence and Curl of Electrostatic Fields
\subsection{Div and Curl of E Fields}
The equations given in the previous section are sufficient to compute an electrostatic field given an arbitrary charge distribution; however, these integrals can be formidable. We now introduce techniques to simplify the computation of such fields, primarily via applications of Gauss' law:
\eqn{\oint \vec{E} \cdot d\vec{S} = \frac{Q_{\text{enc}}}{\epsilon_0} 
\quad \iff \quad \vec{\del}\cdot\vec{E} = \frac{\rho}{\epsilon_0}
}
% 	Gauss' Law / Field Lines / Flux
% 	Divergence of E / Apply Gauss' Law
% 	Curl of E
% Electric Potential
\subsection{Electric Potential}
% 	Definitions of Potential
% 	Poisson's Equation / Laplace Equation
% 	Boundary Conditions
% Work and Energy in electrostatics
\subsection{Work and Energy in Electrostatics}
% 	Charge configuration energy
% 	Electrostatic Energy
% Conductors
\subsection{Conductors}
% 	Basic Properties
% 	Induced Charges
% 	Surface Charges
% 	Capacitors


\section{Potentials}
% Laplace Equation
% 	Introduction
%	Laplace Equation in One Dimension
%	Laplace Equation in Two Dimensions
%	Laplace Equation in Three Dimensions
% 	Boundary Conditions / First Uniqueness
% 	Conductors / Second Uniqueness 
% Method of Images
% 	Classic Problem
% 	Induced Surface Charge
% 	Force and Energy
% 	Other Image Problems
% Separation of Variables
% 	Cartesian Coordinates
% 	Spherical Coordinates
% Multipole Expansion
% 	Approximate Potentials
% 	Monopole and Dipole Terms
% 	Origin of Coordinates in Expansion
% 	E Field of Dipole


\section{Electric Fields in Matter}
\section{Magnetostatics}
\section{Magnetic Fields in Matter}
\section{Electrodynamics}
\section{Conservation Laws}
\section{Electromagnetic Waves}
\section{Potentials and Fields}
\section{Radiation}
\section{Circuits}

