\newpage
%%%%%%%%%%%%%%%%%%%%%%%%%%%%%%%%%%%%%%%%%
%        ELECTRODYNAMICS SECTION        %
%%%%%%%%%%%%%%%%%%%%%%%%%%%%%%%%%%%%%%%%%
\chapter{Electricity \& Magnetism}
\label{sec:electro}
The electricity and magnetism subject matter in the qualifying exam is primarily at the level of \cite{griffithsIntroductionElectrodynamics2018}, a common, upper-level undergraduate electricity and magnetism text.


\section{Electrostatics}
We begin with the study of static electric fields. The electric field is one of two principle elements in electrodynamics, 
% Electric Field
\subsection{Electric Field}
% 	Superposition
The \textit{Principle of Superposition} in electrostatics states that the forces between two charges do not depend on any other charges. Consequently, multiple such forces and fields may be \textit{superimposed}, or added together, to construct the net force and field.
% 	Coulomb Law
Now that we know we can add forces between pairs of charges together, we must describe the force between two electric charges, \textit{Coulomb's law}:
\eqn{\vec{F} = k\frac{qQ}{r^2}\vec{\hat{r}} 
\quad\quad \text{where} \quad\quad k = \frac{1}{4\pi \epsilon_0} 
\quad\quad \text{and} \quad\quad \vec{r}=\vec{r}_Q - \vec{r}_q} 
In the above, $\epsilon$ is the \textit{permittivity of free space}. Note that for like sign charges, $\vec{F}$ is repulsive, and for dislike sign charges, $\vec{F}$ is attractive.
% 	Electric Field
The above equations also make use of the \textit{test charge} $Q$, but it is also possible to discuss the vector field $\vec{E}$ created by the source charges $q$ by the following:
\eqn{\vec{F} = Q\vec{E} 
\quad\quad \text{where} \quad\quad 
\vec{E}(\vec{r}) = k \sum_i \frac{q_i}{r_i^2}\vec{\hat{r}}_i = k \int \frac{1}{r^2}\vec{\hat{r}}\ dq,
\quad \vec{r}_i = \vec{r} - \vec{r}_{q_i}}
% 	Continuous Charge Distribution
For continuous charge distributions, the above can be rewritten in terms of various charge \textit{densities}. In the linear case, a line charge density $\lambda(\vec{r})$ is used. Similarly, for surface charges, $\sigma(\vec{r})$ is used, and for volume charges $\rho(\vec{r})$ is used.
\eqn{\begin{aligned}
	\vec{E}(\vec{r}) &= k \int \frac{\lambda\wrap{\vec{r}'}}{r^2}\vec{\hat{r}}\ dl' \\
	\vec{E}(\vec{r}) &= k \int \frac{\sigma\wrap{\vec{r}'}}{r^2}\vec{\hat{r}}\ dS' \\
	\vec{E}(\vec{r}) &= k \int \frac{\rho\wrap{\vec{r}'}}{r^2}\vec{\hat{r}}\ d\tau' \\
\end{aligned}}
% Divergence and Curl of Electrostatic Fields
\subsection{Div and Curl of E Fields}
The equations given in the previous section are sufficient to compute an electrostatic field given an arbitrary charge distribution; however, these integrals can be formidable. We now introduce techniques to simplify the computation of such fields, primarily via applications of Gauss' law:
% 	Gauss' Law / Field Lines / Flux
\eqn{\oint \vec{E} \cdot d\vec{S} = \frac{Q_{\text{enc}}}{\epsilon_0} 
\quad \iff \quad \vec{\del}\cdot\vec{E} = \frac{\rho}{\epsilon_0}
}
% 	Divergence of E / Apply Gauss' Law
Gauss' law can be applied in cases in which $\rho$ exhibits symmetry, such that a \textit{Gaussian surface} can be constructed containing the charge. On such a surface, $\vec{E}\cdot d\vec{S}$ must be constant, so that it can be factored out of the integrand. For instance, consider the field generated by a sphere of radius $R$ and charge $q$, at a point $r>R$.
\eqn{\int\limits_{\mathcal{S}} \norm{\vec{E}} \cdot da = \norm{\vec{E}} \int\limits_{\mathcal{S}}da = \norm{\vec{E}} 4\pi r^2 \implies \vec{E} = k\frac{q}{r^2}\vec{\hat{r}}}
% 	Curl of E
Similar methods can be used for an infinite sheet of charge, to produce $E=\frac{\sigma}{2\epsilon_0}$. We've exploited the divergence of the $\vec{E}$ field thus far; now we turn the the curl of an $\vec{E}$ field. Starting with a simple example of a point charge, one obtains $\int_a^b \vec{E} \cdot d\vec{l} = kq \wrap{1/r_a - 1/r_b}$, which only depends upon the end points of the integration. Consequently, $\vec{E}$ is a conservative field, and we have:
\eqn{\oint \vec{E} \cdot d\vec{l} = 0 \iff \vec{\del}\times \vec{E} = 0}
% Electric Potential
\subsection{Electric Potential}	
Since the curl of $\vec{E}$ is zero, is it a closed form, and is therefore also an exact form. We know that exact forms are the exterior derivatives of simpler forms, and therefore that $\vec{E} = -\del V$, where $V$ is called the \textit{scalar potential}. For more on closed and exact forms, see Appendix \ref{app:vector-analysis:potentials}. 
% 	Definitions of Potential
We can then define the potential directly:
\eqn{V(\vec{r}) \equiv - \int_{\mathcal{O}}^{\vec{r}} \vec{E} \cdot d\vec{l} 
\quad\quad\quad V(\vec{r}) = - \int_{\infty}^{\vec{r}} k \frac{\rho(\vec{r}')}{r'}d\tau'}
% 	Poisson's Equation / Laplace Equation
The choice of reference point $\mathcal{O}$ amounts to a the addition of a constant to $V$. The conventional choice is \textit{infinitely far away}, or $\mathcal{O}=\infty$. Combining both $\vec{\del}\cdot \vec{E}$ and $\vec{E}=-\del V$, we obtain the general \textit{Poisson} equation, which is the \textit{Laplace} equation.
\eqn{\del^2 V = -\frac{\rho}{\epsilon_0}\quad \text{(Poisson)}
\quad\quad\quad\quad\quad \rho\to 0 \implies \del^2 V = 0 \quad \text{(Laplace)}}
% 	Boundary Conditions
From the infinite-sheet example we can infer important details about continuity of electric fields when a boundary is involved. A charged boundary will induce a discontinuity in the perpendicular component $E^{\perp}$, whereas the parallel component $\vec{E}^{\parallel}$ is always continuous. Specifically:
\eqn{E^{\perp}_{\text{ext}} - E^{\perp}_{\text{int}} = \frac{\sigma}{\epsilon_0} \quad\quad\quad \vec{E}^{\parallel}_{\text{ext}} = \vec{E}^{\parallel}_{\text{int}}}
It should be noted that the potential $V$ is \textit{always} continuous, whereas the gradient of the potential is not necessarily continuous, since $-\vec{E} = \del V$.
% Work and Energy in electrostatics
\subsection{Work and Energy in Electrostatics}
We have discussed field configurations and how those fields give rise to forces on test charges. Now, we discuss the work and energy induced by the electric force, using the conventional, Newtonian definition of work:
\eqn{W = \int_a^b \vec{F} \cdot d\vec{l} = -Q \int_a^b \vec{E} \cdot d\vec{l} = Q\wraph{V\wrap{\vec{b}} - V\wrap{\vec{a}}}}
% 	Charge configuration energy
Note that the energy of a charge configuration may be computed by evaluating the work required to move each point charge from infinity to its final location: $W_i = q_i V(\vec{r}_i)$, since we choose $\vec{V}_{\infty}=0$.
When configuring multiple charges, the potential will change after each charge has been put in place, and will affect subsequent charges. For example, in a system of three charges, the first requires no work $W_1=0$, the second is only affected by the first $W_2 = k q_2 q_1 / r_{12}$, but the third is affected by both the first two charges $W_3 = k q_3 \wrap{q_1 / r_{13} + q_2 / r_{23}}$. For a general system of $n$ charges:
\eqn{W = k \sum_{i=1}^n \sum_{j>i}^n \frac{q_i q_j}{r_{ij}} \quad\quad \text{where} \quad r_{ij} = \norm{\vec{r}_j - \vec{r}_i}}
% 	Electrostatic Energy
% Conductors
\subsection{Conductors}
% 	Basic Properties
% 	Induced Charges
% 	Surface Charges
% 	Capacitors


\section{Potentials}
% Laplace Equation
% 	Introduction
%	Laplace Equation in One Dimension
%	Laplace Equation in Two Dimensions
%	Laplace Equation in Three Dimensions
% 	Boundary Conditions / First Uniqueness
% 	Conductors / Second Uniqueness 
% Method of Images
% 	Classic Problem
% 	Induced Surface Charge
% 	Force and Energy
% 	Other Image Problems
% Separation of Variables
% 	Cartesian Coordinates
% 	Spherical Coordinates
% Multipole Expansion
% 	Approximate Potentials
% 	Monopole and Dipole Terms
% 	Origin of Coordinates in Expansion
% 	E Field of Dipole


\section{Electric Fields in Matter}
\section{Magnetostatics}
\section{Magnetic Fields in Matter}
\section{Electrodynamics}
\section{Conservation Laws}
\section{Electromagnetic Waves}
\section{Potentials and Fields}
\section{Radiation}
\section{Circuits}

