\newpage
%%%%%%%%%%%%%%%%%%%%%%%%%%%%%%%%%%%%%%%%%
%       SPECIAL RELATIVITY SECTION      %
%%%%%%%%%%%%%%%%%%%%%%%%%%%%%%%%%%%%%%%%%
\chapter{Special Relativity}
\label{sec:specrel}
The special relativity subject matter in the qualifying exam is primarily at the level of \cite{ohanianModernPhysics1995}, a common, upper-level undergraduate modern physics text. The main innovation in special relativity is Einstein's second postulate:
\quotebox{
\begin{enumerate}[label=\Roman*.]
	\item Only the relative motion of inertial frames can be measured; the laws of physics are the same in all inertial reference frames. The concept of "absolute rest" is meaningless.
	\item The velocity of light is a universal constant, independent of any relative motion of the source and observer
\end{enumerate}}

\section{Lorentz Invariance}
% Galilean Invariance
Newtonian mechanics, which underlies all of classical mechanics, assumes a universal background space and time, such that the time is shared by all inertial reference frames. Thus a transformation from one Newtonian frame $K$ to another $K'$, separated by velocity $v$ in the $x_1$ dimension, called a \textit{Galilean} transformation takes the form $x_1'=x_1-vt$, $x_2'=x_2$, $x_3'=x_3$, $t'=t$. The fact that Newtonian mechanics is invariant under these transformations is called \textit{Galilean invariance}.\\
% Lorentz Transforms
\indent Galilean invariance isn't compatible with the second postulate of special relativity, so the symmetries of spacetime in special relativity must be different. The \textit{Lorentz transformation} can be derived from the second postulate. Keeping the speed of light $c$ constant, and computing the distance traveled by a pulse of light in two frames, $K$ and $K'$ separated by a boost of speed $v$ on the $x_1$-axis the resulting transformation is defined:
\eqn{\begin{split}
	x_1' &= \gamma\wrap{x_1 - vt} \\
	x_2' &= x_2 \\
	x_3' &= x_3 \\
	t' &= \gamma\wrap{t - x_1}
\end{split}\ \Vast{\}}
	\quad\quad\quad\quad
\gamma=\frac{1}{\sqrt{1-\wrap{\frac{v}{c}}^2}}
}
% Lorentz Group (boosts + rotations)
Any physical system preserved under Lorentz transformations is said to be \textit{Lorentz invariant}. Maxwell's equations of electrodynamics possess Lorentz invariance, and were a major source of motivation for the development of special relativity.

