\newpage
%%%%%%%%%%%%%%%%%%%%%%%%%%%%%%%%%%%%%%%%%
%       SPECIAL RELATIVITY SECTION      %
%%%%%%%%%%%%%%%%%%%%%%%%%%%%%%%%%%%%%%%%%
\chapter{Special Relativity}
\label{sec:specrel}
The special relativity subject matter in the qualifying exam is primarily at the level of \cite{ohanianModernPhysics1995}, a common, upper-level undergraduate modern physics text. The main innovation in special relativity is Einstein's second postulate:
\quotebox{
\begin{enumerate}[label=\Roman*.]
	\item Only the relative motion of inertial frames can be measured; the laws of physics are the same in all inertial reference frames. The concept of "absolute rest" is meaningless.
	\item The velocity of light is a universal constant, independent of any relative motion of the source and observer
\end{enumerate}}

\section{Lorentz Invariance}
% Galilean Invariance
Newtonian mechanics, which underlies all of classical mechanics, assumes a universal background space and time, such that the time is shared by all inertial reference frames. Thus a transformation from one Newtonian frame $K$ to another $K'$, separated by velocity $v$ in the $x_1$ dimension, called a \textit{Galilean} transformation takes the form $x_1'=x_1-vt$, $x_2'=x_2$, $x_3'=x_3$, $t'=t$. The fact that Newtonian mechanics is invariant under these transformations is called \textit{Galilean invariance}.\\
% Lorentz Transforms
\indent Galilean invariance isn't compatible with the second postulate of special relativity, so the symmetries of spacetime in special relativity must be different. The \textit{Lorentz transformation} can be derived from the second postulate. Keeping the speed of light $c$ constant, and computing the distance traveled by a pulse of light in two frames, $K$ and $K'$ separated by a boost of speed $v$ on the $x_1$-axis the resulting transformation is defined:
\eqn{\begin{split}
	x_1' &= \gamma\wrap{x_1 - vt} \\
	x_2' &= x_2 \\
	x_3' &= x_3 \\
	t' &= \gamma\wrap{t - x_1}
\end{split}\ \Vast{\}}
	\quad\quad\quad\quad
\gamma=\frac{1}{\sqrt{1-\wrap{\frac{v}{c}}^2}}
}
% Lorentz Group (boosts + rotations)
Any physical system preserved under Lorentz transformations is said to be \textit{Lorentz invariant}. Maxwell's equations of electrodynamics possess Lorentz invariance, and were a major source of motivation for the development of special relativity.


\section{Space-Time and Four-Vectors}
% 4-vectors
Since special relativity deals with transformations in 4-dimensional spacetime, it is helpful to introduce the concept more formally. We adopt the concept of \textit{four-vectors}, such as $x^\mu$ to represent a point in spacetime, where $x^0=ct$, $x^1=x$, $x^2=y$, $x^3=z$. We also define the square of the spacetime interval between two points to be an invariant:
\eqn{\Delta s^2 = -\wrap{c\Delta t}^2 + \wrap{\Delta x}^2 + \wrap{\Delta y}^2 + \wrap{\Delta z}^2}
We can also define the invariant in terms of the \textit{Minkowski metric} $\eta_{\mu\nu}$ which defines a flat Lorentzian manifold. Further, we adopt the \textit{Einstein summation notation} whereby the $\sum$ symbol is omitted in favor of matching a lower and upper index. For instance, computing the invariant interval for two points $x^\mu$ and $x'^\mu$, where $\Delta x^\mu = \wrap{x^0-x'^0, x^1-x'^1, x^2-x'^2, x^3-x'^3}$.
\eqn{\eta_{\mu\nu}=\begin{pmatrix}
	-1 & 0 & 0 & 0 \\
	0 & 1 & 0 & 0 \\
	0 & 0 & 1 & 0 \\
	0 & 0 & 0 & 1 
\end{pmatrix}\quad\quad\quad
\Delta s^2=\eta_{\mu\nu}\wrap{\Delta x}^\mu\wrap{\Delta x}^\nu}
% Metric, Causality
At this point it's worth noting that Lorentz transformations $\Lambda$ are those that leave the metric unchanged $\Lambda \eta \Lambda^{T}$, which implies that $\Lambda^T = \Lambda^{-1}$ or that the Lorentz transformations are \textit{orthogonal}. We define causal relationships between two points $x^\mu$ and $x'^\mu$ in the following way:
\eqn{\begin{align}
	\Delta s^2 &< 0\quad \text{\textit{timelike}, possible causal dependence} \\
	\Delta s^2 &= 0\quad \text{\textit{lightlike}, boundary between causal dependence and independence} \\
	\Delta s^2 &> 0\quad \text{\textit{spacelike}, causal independence} \\
\end{align}}
For the case of lightlike separation, we define the \textit{proper time} $\Delta \tau$ to be the negative of the interval such that $\Delta \tau = - \Delta s$. For a parametric path $x^\mu(\lambda)$ in spacetime such that the velocity is timelike at every point, the proper time defines the time experienced by an observer moving along the path:
\eqn{\tau=\int_{\lambda_0}^{\lambda_1}\sqrt{-\eta_{\mu\nu}\frac{dx^\mu}{d\lambda}\frac{dx^\nu}{d\lambda}}d\lambda}

