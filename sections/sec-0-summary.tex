\newpage
%%%%%%%%%%%%%%%%%%%%%%%%%%%%%%%%%%%%%%%%%
%            SUMMARY SECTION            %
%%%%%%%%%%%%%%%%%%%%%%%%%%%%%%%%%%%%%%%%%
\chapter{Summary}
\label{sec:summary}

% add note about "this isn't to teach you the material for the first time!!!" only use to review

The Qualifying Examination is given twice a year near the
beginning of the spring and fall semesters \cite{whiteheadGraduateHandbookPenn2014}. 
The exam is split into two three-hour sessions (morning and afternoon), and is composed of a total of about eight questions covering the core areas of physics traditionally included in the undergraduate curriculum:

\begin{enumerate}
	\item Classical Mechanics
	\item Special Relativity
	\item Electricity and Magnetism
	\item Quantum Mechanics
	\item Statistical Mechanics and Thermal Physics
\end{enumerate}

The exam tests fundamental understanding and mastery of basic physical concepts.
The level of knowledge needed is primarily that of undergraduate physics at all levels, but with a level of performance appropriate for a graduate student. 
A detailed description of the examination including sample questions and a list of representative textbooks is available from the Graduate coordinator; previous exams are also on the department’s website. All entering students are required to take the examination when it is given in the fall semester. A student who performs appropriately well on this examination will be considered to have satisfied the examination portion of the procedure for admission to candidacy. All other firstyear students will need to take the examination in the spring semester; their performance on the first examination will be ignored in the candidacy decision.
 
