\newpage
%%%%%%%%%%%%%%%%%%%%%%%%%%%%%%%%%%%%%%%%%
%      CLASSICAL MECHANICS SECTION      %
%%%%%%%%%%%%%%%%%%%%%%%%%%%%%%%%%%%%%%%%%
\chapter{Classical Mechanics}
\label{sec:classmech}
The classical mechanics subject matter in the qualifying exam is primarily at the level of \cite{thorntonClassicalDynamicsParticles2004}, a common, upper-level undergraduate mechanics text.


%%%%%%%%%%%%%%%%%%%%%%%%%%%%%%%%%%%%%%%%%
%          NEWTONIAN MECHANICS          %
%%%%%%%%%%%%%%%%%%%%%%%%%%%%%%%%%%%%%%%%%
\section{Newtonian Mechanics}
\label{sec:cm-newtonian}
Newtonian mechanics is typically covered in the introductory-level mechanics courses, but the basics are worth revisiting for completeness. \index{Newton}

Objects do not change their motion unless acted on by external forces, and when acted upon, the acceleration is linearly proportional to the net applied force. The coefficient is \textbf{inertial mass}, not to be confused with other notions of "mass" as will be seen in Sec \ref{sec:specrel}

\eqn{\sum \vec{F} = \vec{F}_{net} = m\vec{a}}

The solution for Newtonian equations of motion under constant acceleration, for a general coordinate $q$ is given by: 

\eqn[eq:cm-newt-coord]{q(t) = q_0 + \dot{q}t + \frac{1}{2}\ddot{q}t^2}

In the case of a scalar potential $U$, the force vector $\vec{F}$ is derived as the negative gradient of the potential:
\eqn{\vec{F} = -\del U}

Also worth mentioning, the work-energy theorem relates the force vector $\vec{F}$ along a curve $C$ to the change in potential energy. Be careful to get the signs correct in $\vec{F}$ and $\vec{s}$ before the dot product to get the right signs in the result.
\eqn{W = \int_C\vec{F}(\vec{s})\cdot d\vec{s} = \Delta U} \index{Work-Energy Theorem}

In the absence of external forces, the linear momentum $\vec{p} = m\vec{v}$ is conserved, as is the total energy of the system $E = T + U$, where $T = \frac{1}{2}mv^2$ is the Newtonian kinetic energy and $U$ is the potential energy. \index{Conservation of momentum}
\eqn{\ddt[\vec{p}] = \ddt[E] = 0}


%%%%%%%%%%%%%%%%%%%%%%%%%%%%%%%%%%%%%%%%%
%            ANGULAR MOTION             %
%%%%%%%%%%%%%%%%%%%%%%%%%%%%%%%%%%%%%%%%%
\section{Angular Motion}
\label{sec:cm-angular}
\index{Rotation}
Another topic that should be thoroughly engrained in all introductory undergraduate mechanics courses, angular motion will undoubtedly appear on the qualifying exam. We briefly review some key concepts from rotational motion here.

Most of the equations are similar from Sec \ref{sec:cm-newtonian}, with the substitution from translational position coordinates to rotational ones $x \rightarrow \theta$, $v\rightarrow \omega$, etc. Note that Eq \ref{eq:cm-newt-coord} still holds for $q=\theta$. Nearly all the Cartesian-coordinate equations of Newtonian mechanics have a spherical-coordinate analog. For instance $v = \dot{x}$ becomes $\omega = \dot{\theta}$.


The tangential velocity can be related to the angular velocity and radial vector, as can the angular momentum to the linear momentum. Newton's laws about net force and acceleration also have a direct, angular analogue, as does translational kinetic energy:
\eqn{\vec{v}_{\perp} = \vec{\omega}\times \vec{r}}

\eqn{\vec{L} = I \vec{\omega} = \vec{r}\times\vec{p}\where I = \sum_i m_i r_i^2 = \int r^2 dm}

\eqn{\sum\vec{\tau}=\vec{\tau}_{net}=I\ddot{\theta}=\ddt[\vec{L}] \where \vec{\tau} = \vec{r}\times\vec{F}}

\eqn{T_{rot} = \frac{1}{2}I\omega^2}

\index{Angular Momentum}
The moment of inertia is a nuanced mathematical object, given proper treatment in Sec \ref{sec:cm-rigid}. In the simpler cases it is helpful knowing some common moments of inertia:

\eqn{I_{\mathrm{sphere}} = \frac{2}{5}MR^2,\quad I_{\mathrm{shell}} = \frac{2}{3}MR^2}
\eqn{I_{\mathrm{disk}} = \frac{1}{2}MR^2,\quad I_{\mathrm{hoop}} = MR^2}
\eqn{\label{eq:cm-I-rod} I_{\mathrm{rod,end}} = \frac{1}{12}ML^2}

The parallel axis theorem states that if an object has moment of inertia $I_{\hat{a}}$ about an axis $\hat{a}$, and a moment of inertia $I_{\hat{b}}$ about an axis $\hat{b}$, then: 
\eqn{I_{\hat{b}} = I_{\hat{a}} + Md^2\where \hat{b} \parallel \hat{a} \mathrm{\ and\ is\ separated\ by\ distance\ } d}
\index{Parallel-Axis Theorem}

Note that the above can be used to derive $I_{\mathrm{rod,center}} = \frac{1}{3}ML^2$ from Eq \ref{eq:cm-I-rod} with $d = \frac{1}{2}L$.\\

There are also a handful of useful statements about uniform circular motion. The non-slip condition for rolling is when the center of mass and tangential velocities are equal (in magnitude). The acceleration, called centripetal due to its purely-radial nature, can be described in terms of the velocity and radius:\index{Uniform Circular Motion}
\eqn{v_{CM} = v_{\perp}}


\eqn{a_c = \frac{v^2}{r}}


Similar to the Cartesian Newtonian case, in the absence of external torque ($\vec{\tau}_{net}=0$), then the angular momentum is preserved.


%%%%%%%%%%%%%%%%%%%%%%%%%%%%%%%%%%%%%%%%%
%              OSCILLATION              %
%%%%%%%%%%%%%%%%%%%%%%%%%%%%%%%%%%%%%%%%%
\newpage
\section{Oscillation}
\label{sec:cm-oscillation}
\index{Oscillation}
% Harmonic Motion
\subsection{Harmonic Motion}
Most treatments of oscillation begin with Hooke's Law (Eq \ref{eq:cm-hooke}) instead of the general wave equation because it's easier to work in one-dimension. That approach leads to the second-order homogeneous equation Eq \ref{eq:cm-sho-de}, with the familiar solutions given by Eq \ref{eq:cm-sho-soln} (complex roots of characteristic equation + Euler's Identity $e^{i\theta} = \cos{\theta} + i\sin{\theta})$. Plugging into $T = \frac{1}{2}m\dot{x}^2$ and $U = \frac{1}{2}kx^2$ gives Eq \ref{eq:cm-sho-energy}.

\eqn{\label{eq:cm-hooke} F(x) = -kx,\quad U = \frac{1}{2}kx^2}\index{Hooke's Law}
\eqn{\label{eq:cm-sho-de} \ddot{x} +  \omega_0^2x = 0\where \omega_0 = \sqrt\frac{k}{m} = 2\pi \nu_0}

\eqn{\label{eq:cm-sho-soln} x(t) = A\sin{\left(\omega_0 t - \delta\right)}}

\eqn{\label{eq:cm-sho-energy} E = T + U = \frac{1}{2}kA^2}


% Damped Oscillation
\subsection{Damped Oscillation}
When a damping force is added like Eq \ref{eq:cm-damp-force}, the equation of motion has an extra term \index{Oscillation!Damped}
\eqn{\label{eq:cm-damp-force} F_{\mathrm{damp}} = -b\dot{x}}
\eqn{\label{eq:cm-sho-de} \ddot{x} + 2\beta \dot{x} + \omega_0^2x = 0\where \beta = \frac{b}{2m}}
\eqn{\label{eq:cm-dho-soln}
	x(t) = e^{-\beta t}\left[A_1 e^{i\omega t} + A_2e^{-i\omega t}\right]\where \omega^2=\omega_0^2-\beta^2}

Note in the above solution Eq \ref{eq:cm-dho-soln}, the relative values of $\omega_0$ and $\beta$ can produce phenomenologically distinct behavior. Specifically, the cases are:
\begin{subequations}
\begin{align}
		\omega^2 &> 0 & \omega_0^2 &> \beta^2 & \text{Underdamped} \\
		\omega^2 &< 0 & \omega_0^2 &< \beta^2 & \text{Overdamped} \\
		\omega^2 &= 0 & \omega_0^2 &= \beta^2 & \text{Critically Damped} 
\end{align}	
\end{subequations}
Recall the general behavior for the above solution categories: $(i)$ underdamped solutions show an exponentially-decaying oscillation, and will periodically cross the x axis, $(ii)$ overdamped solutions are slowly decaying exponentials without oscillation, and $(iii)$ critically-damped is the unique, most-rapidly decaying exponential without oscillation.\\


% Driven Oscillation
\subsection{Driven Oscillation}
In the case of a driving force (sinusoidal for simplicity), the equation of motion becomes inhomogeneous, and we must solve for the particular solution $x_p(t)$ or "steady-state" effects (as opposed to the complementary solution $x_c(t)$ or "transient" effects of the homogeneous equation given by Eq \ref{eq:cm-dho-soln}). The steady state effects are dominant for $t >> 1/\beta$. \index{Oscillation!Driven}\index{Waves!Transient}\index{Waves!Stead-state}

\eqn{\ddot{x} + 2\beta \dot{x} + \omega_0^2x = A\cos{\omega t}\where A = \frac{F_0}{m}}
\eqn{x_p(t) = D\cos{\left(\omega t - \delta\right)}\where D=\frac{A}{\sqrt{\left(\omega_0^2 - \omega^2\right)^2 + 4\omega^2\beta^2}},\quad \delta=\tan^{-1}{\left(\frac{2\omega\beta}{\omega_0^2 - \omega^2}\right)}}

Resonance effects can be computed by finding the angular frequency $\omega_R$ at which the amplitude $D$ is maximized:
\eqn{\omega_R = \sqrt{\omega_0^2 - 2\beta^2}\where \frac{dD}{d\omega}\bigg{\rvert}_{\omega=\omega_R}=0}
Note that for weak damping, or where $\beta \rightarrow 0$, then the resonant frequency approaches the natural frequency of the system $\omega_R \rightarrow \omega_0$. \index{Resonance}
\\
% Response functions (omitted since optional, may revisit)


%%%%%%%%%%%%%%%%%%%%%%%%%%%%%%%%%%%%%%%%%
%          LAGRANGIAN MECHANICS         %
%%%%%%%%%%%%%%%%%%%%%%%%%%%%%%%%%%%%%%%%%
\newpage
\section{Lagrangian Mechanics}
\label{sec:cm-lagrangian}

Lagrangian Mechanics marks the separation between introductory and upper-level, undergraduate mechanics material. 
\subsection{Principle of Least Action}
Though most courses introduce the Lagrangian as in Eq \ref{eq:cm-lag-def}, there is a deeper physical principle  motivating the choice of $\Lag$, called Hamilton's Principle:\index{Hamilton's Principle} 
\quotebox[0.8]{Of all the possible paths along which a dynamical system may move from one point to another within a specified time interval (consistent with any constraints), the actual path followed is that which minimizes the time integral of the difference between the kinetic and potential energies.}

The above principle, when combined with a rigorous understanding of calculus of variations (included in Appendix \ref{app:calcvar}), explains why the conventional form of the Lagrangian is $T - U$. To paraphrase Hamilton, \textit{nature seeks the path of least action} by converting the least amount of potential energy into kinetic energy. This is formally represented by Eq. \ref{eq:cm:hamilton-principle}, which ultimately leads to the familiar Lagrangian equations of motion (Eq. \ref{eq:cm:lag-eqn}):

\eqn[eq:cm-lag-def]{\Lag = T - U}
\eqn[eq:cm:hamilton-principle]{\delta \int_{t_1}^{t_2}\left(T - U\right)dt = 0}
\eqn[eq:cm:lag-eqn]{\pd{\Lag}{q_i} - \ddt \pd{\Lag}{\dot{q}_i} = 0 \where p_i \equiv \pd{\Lag}{\dot{q_i}}}\

% Generalized coordinates
\subsection{Generalized Coordinates}
The above makes use of \textit{generalized coordinates}, denoted by $q_i$, which are a set of $s$ independent parameters used to fully describe the state of the system. For instance, in the case of $n$ particles on 3-space, $s = 3n$. When constraints are imposed, the number of independent parameters is reduced by the number $m$ of constraints, so $s = 3n - m$. The \textit{generalized velocities} are the time derivatives of the generalized coordinates, $\dot{q}_1, \dot{q}_2, ..., \dot{q}_s$. Transforming between Cartesian and generalized coordinates can be expressed as:
\begin{equation}
\begin{split}
		x_{\alpha,i} &= x_{\alpha, i}(q_j, t) \\
		\dot{x}_{\alpha,i} &= \dot{x}_{\alpha, i}(q_j, \dot{q}_j, t) 
\end{split}	
\end{equation}

Where $\alpha = 1, 2, ..., n$, $i = 1, 2, 3$, and $j = 1, 2, ..., s$. The inverse transformations may be written similarly. If there are $k = 1, 2, ..., m$ constraints, they may also be written as in Eq \ref{eq:cm:lag-const-m}. Constraints of this form are called \textit{holonomic}. They interrelate  conventional coordinates and possibly time.
\begin{equation}
\begin{split}
		q_{j} &= q_{j}(x_{\alpha, i}, t) \\
		\dot{q}_{j} &= \dot{q}_j(x_{\alpha, i}, \dot{x}_{\alpha, i}, t) 
\end{split}	
\end{equation}

\eqn[eq:cm:lag-const-m]{f_k(x_{\alpha, i}, t) = 0}


% Undetermined Multipliers
\subsection{Undetermined Multipliers}
If all the constraints in a system are holonomic, then is always possible to find a proper set of generalized coordinates such that the equations of motion do not reference the constraints. However, if the constraints are expressible in differential form (Eq. \ref{eq:cm:diff-const}), then it is possible to incorporate them directly into the equations of motion as in Eq. \ref{eq:cm:lag-motion-const}.
\eqn[eq:cm:diff-const]{\sum\limits_{j}\pd{f_k}{q_j}dq_j = 0\where k = 1, ..., m\ \text{and} j = 1, ..., s}
\eqn[eq:cm:lag-motion-const]{\pd{\Lag}{q_j} - \ddt\pd{\Lag}{\dot{q}_j} + \sum\limits_{k}\lambda_{k}(t)\pd{f_k}{q_j} = 0}
Where the $\lambda_k(t)$ are the \textit{undetermined Lagrangian multipliers}. These multipliers are closely related to the generalized forces of constraint $Q_j$ by:
\eqn{Q_j = \sum\limits_{k}\lambda_k \pd{f_k}{q_j}}


% Example usage: pendulum
\subsection{Example: Simple Pendulum}
\label{sec:cm-lag-pend}
The below example shows how Lagrangian formalism can be used to simplify the equations of motion using generalized coordinates:

\texfig[0.35]{cm-lag-pend}{A simple pendulum of length $l$ with a mass $m$ suspended at an angle $\theta$.}

%We must first choose coordinates for the single particle of interest - the suspended mass. We can do this in the Cartesian sense of $(x, y)$, but then would have to impose the constraint $x^2 + y^2 - l^2 = 0$. Fortunately, since the constraint is independent of time, there must exist a set of generalized coordinates where the constraint is implicit. We choose polar coordinates $(r, \theta)$ and so that the constraint is simply $\dot{r}\equiv 0$. 
\eqn{\Lag = T - U = \frac{1}{2}I\omega^2 - mgy = \frac{1}{2}ml^2\dot{\theta}^2 - (-mgl\cos\theta)}
\eqn{\pd{\Lag}{\theta} = -mgl\sin\theta,\quad \pd{\Lag}{\dot{\theta}}=ml^2\dot{\theta}}
\eqn{\ddot{\theta} + \frac{g}{l}\sin\theta = 0\quad\quad \omega_0 \approx \sqrt{\frac{g}{l}},\ \text{for small}\ \theta}



%%%%%%%%%%%%%%%%%%%%%%%%%%%%%%%%%%%%%%%%%
%         HAMILTONIAN MECHANICS         %
%%%%%%%%%%%%%%%%%%%%%%%%%%%%%%%%%%%%%%%%%
\newpage
\section{Hamiltonian Mechanics}
\label{sec:cm-hamiltonian}
\index{Hamiltonian!Mechanics}

% Introduction
The Hamiltonian formulation of mechanics is most simply introduced as the Legendre transform of the Lagrangian formulation in the variable $\dot{q}$. The definition of the Hamiltonian (Eq. \ref{eq:cm-damp-force}) is an application of the more general Legendre transformation (Eq. \ref{eq:app-calc-var-legendre}). For more on Legendre Transforms, see Appendix \ref{sec:app-calc-var-legendre}.
\eqn[eq:cm-ham-def]{\Ham(p, q) = \sum_i p_i \dot{q}_i - \Lag \approx T + U}
The equations of motion are then given by the coupled, first-order differential equations:
\eqn{\dot{p}_i = -\pd{\Ham}{q_i},\quad \dot{q}_i = \pd{\Ham}{p_i}}

% Distinction from Lagrangian
\subsection{Comparison with Lagrangian Formalism}
Since Lagrangian and Hamiltonian formulations are equivalent in that the produce identical dynamics, the most important distinction between them is the number and type of equations they produce. The Lagrangian approach produces $N$, second-order differential equations, whereas the Hamiltonian approach produces $2N$, first-order differential equations, as summarized in the below table:

\begin{center}
	\begin{tabular}{lcl}
	 \textbf{Formulation} & \textbf{Number of Equations} & \textbf{Type of Equations} \\
	\hline
	 Lagrangian & $N$ & Second-order \\
	 Hamiltonian & $2N$ & First-order \\
	\end{tabular}
\end{center}

% "Canonical" terminology
It is worth mentioning briefly that Hamilton's equations of motion are called the \textit{canonical} equations of motion, with $p_i$ and $q_i$ being the \textit{canonical variables}. This terminology signifies a preference for Hamilton's formulation, largely based on the nice, symmmetric form of Eq. \ref{eq:cm-ham-def}. There are deeper reasons to prefer the Hamiltonian to the Lagrangian, such as the connection with symplectic forms in the phase space of the system being compatible with the Hamiltonian and canonical transformations, but these reasons are out of scope for this review.


% Conservation theorems
\subsection{Relation to Energy}
The Hamiltonian can also be derived by modifying the Lagrangian to make the time derivative vanish. The form of the Lagrangian's full time derivative is given by Eq \ref{eq:cm-cons-ham-deriv-1}, where we can set the last term equal to 0 if the system is closed. We use Euler's equation (Eq. \ref{eq:cm:lag-eqn}) to substitute for $\pd{\Lag}{q_i}$ in Eq. \ref{eq:cm-cons-ham-deriv-2}, and lastly exploit the product rule to pull the time derivative out in Eq. \ref{eq:cm-cons-ham-deriv-3}. This result shows that the Hamiltonian is, by default, time independent.  


\eqn[eq:cm-cons-ham-deriv-1]{\ddt[\Lag] = \sum\limits_i\pd{\Lag}{q_i}\dot{q}_i + \sum\limits_i\pd{\Lag}{\dot{q}_i}\ddot{q}_i + \pd{\Lag}{t}\where \pd{\Lag}{t}=0\ \text{for a closed system}}
\eqn[eq:cm-cons-ham-deriv-2]{\ddt[\Lag]=\sum\limits_i \dot{q}_i\ddt\pd{\Lag}{\dot{q}_i}+ \sum\limits_i\pd{\Lag}{\dot{q}_i}\ddot{q}_i \where \pd{\Lag}{q_i} = \ddt \pd{\Lag}{\dot{q}_i}}
\eqn[eq:cm-cons-ham-deriv-3]{\ddt[\Lag] - \sum\limits_i\ddt\left(\dot{q}\pd{\Lag}{\dot{q}_i}\right) = \ddt \left(\Lag - \sum\limits_i q_i p_i \right) = 0}
\eqn{\implies \sum\limits_i p_i q_i - \Lag = \Ham = \text{constant}}

\newpage
The Hamiltonian may be constant in time, but showing that this quantity represents the total energy of the system requires a few more steps:
\eqn[eq:cm-cons-ham-1]{\pd{\Lag}{\dot{q}_i} = \pd{\left(T - U\right)}{\dot{q}_i} = \pd{T}{\dot{q}_i}\quad \text{iff}\quad \pd{U}{\dot{q}_i}=0}
\eqn[eq:cm-cons-ham-2]{\Ham = \sum\limits_i \dot{q}_i \pd{T}{\dot{q}_i} - (T - U) = T + U = E\quad \text{iff}\quad\sum\limits_i \dot{q}_i \pd{T}{\dot{q}_i} = 2T}
The Hamiltonian thus is equivalent to the energy of the system under two conditions, represented above in the "if and only if" statements from Eqns \ref{eq:cm-cons-ham-1} and \ref{eq:cm-cons-ham-2} respectively. To state these conditions more explicitly:

\begin{enumerate}
	\item The potential must be velocity independent, thus allowing for the elimination of the $\pd{U}{\dot{q}_i}$ terms.
	\item Coordinate transformation from Cartesian to generalized coordinates must be independent of time, ensuring kinetic energy is a homogeneous quadratic function of the conjugate momenta variables $\dot{q}_i$, e.g. $T = \sum_{i,j}a_{ij}\dot{q}_i\dot{q}_j$.
\end{enumerate}


% Liouville and Hamiltonian Phase Space
\subsection{Liouville's Theorem}
Liouville's Theorem is a statement about the evolution of the density of \textit{similar} states of a system, and is the starting point for statistical mechanics. An example of the power of the Hamiltonian approach to mecahnics, the Liouville Theorem shows that densities of similar states (or volume forms, dually) are unaffected along Hamiltonian flows. Said more in the language of undergraduate mechanics:

\quotebox{The density of states in an ensemble of many identical states with different initial conditions is constant along every trajectory in phase space.}
\eqn{\ddt[\rho] = 0}


% Example usage: pendulum
\subsection{Example: Simple Pendulum Revisited}
The below example shows how the Hamiltonian formalism can be used similarly to how the Lagrangian formalism was used in Sec \ref{sec:cm-lag-pend}.

\eqn{\Ham = \sum\limits_{i}p_i \dot{q}_i - \Lag = p_{\theta}\dot{\theta} - \Lag = \frac{1}{2}ml^2\dot{\theta}^2 - mgl\cos\theta = T + U = E_{\text{total}}}
\eqn{\Ham(\theta, p_{\theta}) = \frac{p_{\theta}^2}{2ml^2} - mgl\cos{\theta}}

\begin{subequations}
\begin{alignat}{3}
	\dot{\theta} &= &\pd{\Ham}{p_{\theta}} &= \frac{p_{\theta}}{ml^2} \\
	\dot{p}_{\theta} &= &-\pd{\Ham}{\theta} &= -mgl\sin{\theta} 
\end{alignat}	
\end{subequations}


\eqn{\ddot{\theta} = \frac{\dot{p}_{\theta}}{ml^2} \implies \ddot{\theta} + \omega_0^2\sin\theta = 0\where \omega_0 \approx \sqrt{\frac{g}{l}},\ \text{for small}\ \theta}



%%%%%%%%%%%%%%%%%%%%%%%%%%%%%%%%%%%%%%%%%
%         CENTRAL FORCE MOTION          %
%%%%%%%%%%%%%%%%%%%%%%%%%%%%%%%%%%%%%%%%%
\newpage
\section{Central Force Motion}
\label{sec:cm-cfm}


\subsection{Two-Body Problem}
% reduced mass
The classical two-body problem, depicted below in both an arbitrary and center-of-mass frames, can be reduced to a one-body problem of the \textit{reduced mass} $\mu$ in a central potential $U(r)$.
\texfig[0.55]{cm-central-two-body}{A simple pendulum of length $l$ with a mass $m$ suspended at an angle $\theta$.}

The Lagrangian of the system can be described by the center of mass and the relative radius between the two masses $\vec{r} = \vec{r_2} - \vec{r_1}$, where the potential is a function of $r = \lvert\vec{r}\rvert$.
\eqn{\Lag=\frac{1}{2}\mu \left\lvert\dot{\vec{r}}\right\rvert^2 - U(r) \where\mu = \frac{m_1 m_2}{m_1 + m_2}\quad \text{and}\ r = \lvert\vec{r}\rvert = \lvert\vec{r_2} - \vec{r_1}\rvert}

% Lagrangian
Since the potential only depends on $r$, the resulting spherical symmetry implies that angular momentum is conserved. In turn, this implies that the radial vector and linear momentum vector are bound to a plane normal to the angular momentum. Thus the problem is two-dimensional, and polar coordinates suffice. 
\eqn{\Lag = \frac{1}{2}\mu\left(\dot{r}^2 + r^2\dot{\theta}^2\right) - U(r)}

% Conserved quantites
Because the Lagrangian is independent of $\theta$, the corresponding momentum $p_{\theta}$ is conserved. This should be familiar from $\dot{p_{\theta}} = \pd{\Lag}{\theta} = 0$. The total energy is also guaranteed to be constant since discussion is limited to nondissipative systems.
\eqn{p_{\theta}\equiv \pd{\Lag}{\dot{\theta}} = \mu r^2 \dot{\theta} = \text{constant} = l}
\eqn[eq:cm-central-energy]{E = T + U = \text{constant} = \half\mu \dot{r}^2 + \half \frac{l^2}{\mu r^2} + U(r)}

% Equations of Motion
When $U(r)$ is specified, it's possible using the above to solve for the resulting dynamics $r(t)$, however it is also possible to solve for the orbital shape $r(\theta)$ by substituting the angle $\theta$ for time according to $d\theta = \frac{d\theta}{dt}\frac{dt}{dr}dr = \frac{\dot{\theta}}{\dot{r}}dr$. The resulting equations of motion are:
\eqn{\theta(r) = \int \frac{\pm \left(l/r^2\right)dr}{\sqrt{2\mu \left(E - U - \frac{l^2}{2\mu r^2}\right)}}}

\eqn{\frac{d^2}{d\theta^2}\left(\frac{1}{r}\right) + \frac{1}{r} = -\frac{\mu r^2}{l^2}F(r)}

\subsection{Central Field Effects}
% Orbital properties (rmin, rmax)
Using the radial equation Eq \ref{eq:cm-central-energy}, the condition $\dot{r}=0$ implies a turning point in the motion. In general, the equation below can have two roots, resulting in a minimum radius $r_{\text{min}}$ and a maximum radius $r_{\text{max}}$. The case of circular motion results in a single root with $r_{\text{min}}= r_{\text{max}}$.

\eqn[eq:cm-central-turn]{\dot{r} = 0 \implies E - U(r) - \frac{l^2}{2\mu r^2} = 0}

% Effective potential
The last term in Eq. \ref{eq:cm-central-turn} has units of energy, and can be interpreted as the \textit{centrifugal potential energy} $U_c \equiv \frac{l^2}{2\mu r^2}$. There is a corresponding "force" (not a traditional force) called the \textit{centrifugal force} associated to the potential, $F_c = -\pd{U_c}{r} = \mu r \dot{\theta}^2$. The centrifugal potential can be combined with the real potential $U(r)$ to create an \textit{effective potential} $V(r)$. Note that this potential is \textit{fictitious} in that it combines the real potential $U(r)$ with the energy term associated with the angular motion about the center of force.
\eqn{V(r) = U(r) + \frac{l^2}{2\mu r^2}}

\texfig[0.5]{cm-central-eff-pot}{The effective potential $V(r)$ for a central force with potential $U(r)=-\frac{k}{r}$. The contributing terms, the real potential $U(r)$ and the angular-motion energy term are in \textcolor{dblue}{blue}. Several sample energies are depicted in \textcolor{dgreen}{green}, with the apside radii $r_i$ plotted along the $r$ axis. The case of $E_3$ represent circular motion, $E_2$ represents elliptical motion, and $E_1$ represents hyperbolic motion.}


\subsection{Kepler's Laws}
% Kepler's parameterization of the problem - eccentricity
We briefly review Kepler's parameterization of planetary orbits, as well as his well-known set of three laws defined in terms of orbital variables. The \textit{eccentricity} $\varepsilon$ and the \textit{latus rectum} $2\alpha$ are:
\eqn{\frac{\alpha}{r} = 1 + \varepsilon \cos{\theta}\where \alpha\equiv \frac{l^2}{\mu k},\quad \varepsilon \equiv \sqrt{q + \frac{2El^2}{\mu k^2}}}

% Orbital morphology
Given an effective potential, different types of orbit solutions exist as depicted in Fig \ref{fig:cm-central-orbit-morph}. For elliptical orbits, major and minor axes can be defined The apsidal distances can be specified in terms of the semi major and minor axes of the ellipsical orbit, which in turn are defined in terms of the Energy of the orbit and reduced mass:
\eqn{a = \frac{\alpha}{1 - \varepsilon^2} = \frac{k}{2\left\lvert E\right\rvert},\quad\quad b = \frac{\alpha}{\sqrt{1 - \varepsilon^2}} = \frac{l}{\sqrt{2\mu\left\lvert E\right\rvert}}}
\eqn{r_{\text{min}} = a(1 - \varepsilon) = \frac{\alpha}{1 + \varepsilon},\quad\quad r_{\text{max}} = a(1 + \varepsilon) = \frac{\alpha}{1 - \varepsilon}}
\texfig[0.6]{cm-central-orbit-morph}{Orbit morphologies defined in terms of eccentricity $\varepsilon$. Bound orbits like ellipses and circles have negative energies, $E_{\text{circle}}=V_{\text{min}}$ and $V_{\text{min}}<E_{\text{ellipse}}<0$ respectively. Parabolas are defined by $E_{\text{parabola}}=0$, and hyperbolas have $E_{\text{hyperbola}}>0$.}


% Kepler's laws
We can now state Kepler Laws formally:

\begin{enumerate}
	\item \textit{Planets move in elliptical orbits about the Sun at one focus}, or formally in the case: $m_1 << m_2$
	\item \textit{The are per unit time swept out by a radius vector from the Sun to a planet is constant}, or formally: $\ddt[A] = \frac{l}{2\mu}$
	\item \textit{The square of a planet's period is proportional to the cube of the major axis of the orbit}, or formally: $\tau^2 = \frac{4\pi^2 \mu}{k}\alpha^3$
\end{enumerate}



\subsection{Orbital Dynamics}
% Orbit transfers - just a few sentences inline eqns only
Orbit transfer (Hohmann) important application. two velocity changes, from inner circular orbit to elliptical transfer orbit, then from transfer ellipse to outer circular orbit. The velocity of the inner circular orbit $v_1 = \sqrt{\frac{k}{mr_1}}$, increased to the elliptical transfer orbit $v_{t1}=\sqrt{\frac{2k}{mr_1}\left(\frac{r_2}{r_1 + r_2}\right)}$.


% Precession
Precession, only occurs if exists deviations from $1/r^2$, like perturbations from other gravitational bodies in the system, or relativistic corrections (add a $1/r^4$ term to the force). The \text{apsidal angle} is the angle between subsequent apsides. The angle between successive perihelions (twice the apsidal angle) is given by $\Delta = \frac{6\pi GM}{ac^2 \left(1-\varepsilon^2\right)}$. Orbit stability can be succintly stated in terms of the effective potential and force:
\eqn{\pd{V}{r}\bigg{\rvert}_{r=\rho} = 0\quad\quad\text{and}\quad\quad \frac{\p^2V}{\p r^2}\bigg{\rvert}_{r=\rho} > 0 \quad\implies\quad\frac{F'\left(\rho\right)}{F\left(\rho\right)} + \frac{3}{\rho} > 0}



%%%%%%%%%%%%%%%%%%%%%%%%%%%%%%%%%%%%%%%%%
%          SYSTEMS OF PARTICLES         %
%%%%%%%%%%%%%%%%%%%%%%%%%%%%%%%%%%%%%%%%%
\newpage
\section{Systems of Particles}
\label{sec:cm-particles}
Many properties of the system can be decomposed in that of the center of mass $\vec{R}$ and that of the system about the center of mass. Some other common variables are the mass of the system $M$, the force of the $\alpha$th particle on the $\beta$th particle, $\vec{f}_{\alpha\beta}$, and the mass and position of the $\alpha$th particle, $m_{\alpha}$ and $\vec{r}_{\alpha}$ respectively. External forces on particle $\alpha$ are denoted $\vec{F}_{\alpha}^{\left(e\right)}$, and the momentum of the system is denoted $\vec{P}$.

% Energy and Momenta
\subsection{Energy and Momenta}

% 	Center of Mass
The center of mass $\vec{R}$ is an important feature of a system of particles, and is the "balance" point for the effects of external forces and torques. 
\eqn{\vec{R} = \frac{1}{M}\sum\limits_{\alpha}m_{\alpha}\vec{r}_{\alpha} = \frac{1}{M}\int\vec{r}\ dm}
% 	Linear Momentum
The center of mass $\vec{R}$ moves like a single particle of mass $M$ acted on by the net eternal force $\vec{F}=\sum_{\alpha}F_{\alpha}^{(e)}$. Consequently, the linear momentum $P$ of the system is the same as a single particle of mass $M$ moving with velocity $\dot{\vec{R}}$. Also note, the \textit{total} linear momentum of the system is conserved and equal to $\vec{P}$ if the net external force is zero.
\eqn{\dot{\vec{P}} = M\ddot{\vec{R}} = \vec{F},\quad\quad \dot{\vec{p}}_{\alpha}=\vec{F}_{\alpha}^{(e)} + \sum\limits_{\beta}\vec{f}_{\alpha\beta}}
% 	Angular Momentum
If the net external torque $\vec{N}^{(e)}$ about a given axis is zero, then the total angular momentum of the system about that axis is conserved. Also note, the total internal torque must vanish if the forces are central, since $\vec{r}_{\alpha\beta}\times \vec{f}_{\alpha\beta} = 0$ for central forces.
\eqn{\vec{L} = \vec{R}\times \vec{P} + \sum\limits_{\alpha}\vec{r}_{\alpha}'\times \vec{p}_{\alpha}',\quad\quad \vec{L}_{\alpha} = \vec{r}_{\alpha}\times \vec{p}_{\alpha}, \quad\quad \vec{\dot{L}} = \sum\limits_{\alpha}\vec{r}_{\alpha}\times \vec{F}_{\alpha}^{(e)} = \vec{N}^{(e)}}
% 	Energy
The total kinetic energy of the system $T$ is equal to the kinetic energy of the center of mass and the kinetic energy of the motion of individual particles relative to the center of mass. Total energy for a conservative system is constant.
\eqn{T = \half MV^2 + \sum\limits_{\alpha}\half m_{\alpha}{v_{\alpha}'}^2}

% Collisions
\subsection{Collisions}
We follow the convention of an incident particle striking a stationary particle in the laboratory (\textbf{LAB}) frame, and transform to the center of mass (\textbf{CM}) frame to exploit conservation laws. Primed quantities, $x'$, refer to the \textbf{CM} frame, $\vec{u}_i$ are initial velocities, $\vec{v}_i$ are final velocities. Since particle 2 is initially at rest, the velocity of the center of mass (and reverse velocity of particle 2 in \textbf{CM} frame) is given by:
\eqn{\vec{V} =\frac{m_1 \vec{u}_1}{m_1 + m_2} = -\vec{u}_2'}

% 	Elastic Collision
\newpage
For \textit{elastic} collisions, masses do not change and kinetic energy is conserved. In the \textbf{CM} frame for elastic collisions, the speeds are equal before and after the collision: $u_1' = v_1',\quad u_2' = v_2'$, moving in exactly opposite directions.
\eqn{\tan{\psi} = \frac{\sin\theta}{\cos\theta + \left(m_1 / m_2\right)}}
Note that this gives $\psi \approx \theta$ in the case that $m_1 << m_2$, and $\psi = \half\theta$ in the case that $m_1 = m_2$. In the latter case, the lab angles $\zeta$ and $\psi$ will always form a right angle, $\zeta + \psi = \pi/2$. The figure below summarizes the geometry of the problem. For elastic collisions, the final kinetic energies can also be described in terms of the inital kinetic energy and the scattering angles:
\eqn{\frac{T_1}{T_0}=\frac{m_1^2}{\wrap{m_1 + m_2}^2}\wraph{\cos{\psi}\ \pm\ \sqrt{\wrap{\frac{m_2}{m_1}}^2 - \sin^2{\psi}}\ }^2,\quad \frac{T_2}{T_0} = 1 - \frac{T_1}{T_0}}
Note that for $m_1=m_2$, then the above may be simplified to $T_1/T_0 = \cos^2{\psi}$ and $T_2/T_0 = \sin^2{\psi}$.\\
\texfig[0.75]{cm-syspart-elastic}{Geometry of an elastic collision in both the LAB and CM frames.}\\
% 	Inelastic Collision
For \textit{inelastic} collisions, some energy is lost. This can be either frictional forces (like internal heating of the particles), or it may be converted to mass-energy (like a nuclear collision). The efficient of a collision can be measured by the \textit{coefficient of restitution}, given by Eq. \ref{eq:cm-syspart-coef-rest}. For perfectly elastic collisions, $\varepsilon = 1$, whereas for perfectly inelastic collisions, $\varepsilon = 0$.
\eqn[eq:cm-syspart-coef-rest]{\varepsilon = \frac{\norm{v_2 - v_1}}{\norm{u_2 - u_1}}}
During a collision, an impulse is generated by the force of interaction between the two bodies. This \textit{impulse} is represented by the change in momentum $\vec{P}$ from the beginning of the collision $t_1$ to the end of the collision $t_2$:
\eqn{\vec{F} = \ddt[\vec{p}],\quad\quad \int_{t_1}^{t_2}\vec{F}dt\equiv \vec{P}}


% Scattering and cross sections
\subsection{Scattering and Cross Sections}
% impact parameter
% intensity / flux density
% differential scattering cross section
%
The above can be used to predict the scattering angle when combined with a specific form of the potential $U(r)$. For the below, we assume a repulsive force between $m_1$ and $m_2$ akin to electrostatic repulsion of like charges. The perpendicular offset of the incident particle, the \textit{impact parameter} $b$, is sufficient to determine the scattering angle $\theta$ and uniquely determines the angular momentum of the incident mass about $m_2$, $l=m_1 u_1 b = b\sqrt{2m_1 T_0}$.
\texfig[0.85]{cm-syspart-scatter}{A scattering diagram in both the LAB and CM frames. The $m_1$ mass is incident upon a stationary $m_2$ with an impact parameter $b$, and scatters with angle $\theta$.}\\
Because the impact parameter is impossible to measure directly, we analyze probabilities of scattering angles with various impact parameters. A beam of \textit{intensity} or \textit{flux density} $I$ has $I$ particles passing through per unit time per unit area normal to the beam. We then define the \textit{differential scattering cross section} $\sigma(\theta)$ in the CM frame as the fraction of incident particles per unit area that end up scattered into a solid angle $d\Omega$ at an angle $\theta$:
\eqn{\sigma(\theta) = \frac{d\sigma}{d\Omega} = \frac{1}{I}\frac{dN}{d\Omega}\quad\quad \text{units: } m^2/rad^2 = \text{Area}}
After integrating the azimuthal angle (axial symmetry), and using CM frame for simplification, the cross section can be rewritten in terms of $db/d\theta$, which can be determined from:
\eqn{\sigma(\theta) = \frac{b}{\sin{\theta}}\norm{\frac{db}{d\theta}},\quad \theta = \pi - 2\Theta,\quad \Theta = \int_{r_{\text{min}}}^{\infty}\frac{\wrap{b/r^2}dr}{\sqrt{1 - \wrap{b^2/r^2} - \wrap{U/T_0'}}}}
For converting the CM-Frame scattering angle $\theta$ into the LAB-Frame scattering angle $\psi$, we have:
\eqn{\sigma(\psi) = \sigma(\theta) \cdot \frac{\wraph{x\cos{\psi} + \sqrt{1-x^2\sin^2{\psi}}}^2}{\sqrt{1-x^2\sin^2{\psi}}},\quad\quad\theta=\sin^{-1}{\wrap{x\sin{\psi}}} + \psi}
% 	Rutherford
In the case of Rutherford scattering via electrostatic repulsion, the differential cross section and the \textit{total cross section} $\sigma_t$ can be computed. Note that $\sigma_t$ is infinite because $1/r$ potential doesn't decay fast enough, rather, screening by nearby atoms is sufficient to limit the maximum angle.
\eqn{\sigma(\theta) = \frac{k^2}{\wrap{4T_0'}^2}\cdot\frac{1}{\sin^4{\wrap{\theta/2}}},\quad\quad \sigma_t = \int \sigma(\theta)d\Omega = 2\pi \int_{0}^{\psi_{\text{max}}}\sigma(\psi)\sin{\psi}d\psi}

% \subsection{Rocket Motion} % OMITTED, can add back if essential in test questions
% Rocket Motion



%%%%%%%%%%%%%%%%%%%%%%%%%%%%%%%%%%%%%%%%%
%     NON INERTIAL REFERENCE FRAMES     %
%%%%%%%%%%%%%%%%%%%%%%%%%%%%%%%%%%%%%%%%%
\newpage
\section{Noninertial Reference Frames}
\label{sec:cm-noninertial}

In cases where a local coordinate system undergoes nontrivial motion, it is often easier to use a noninertial reference frame than to attempt the intricacies of the transformation to a fixed frame.
% Rotating coordinate systems
\subsection{Rotating Coordinate System}
Any infinitesimal motion can be represented as a rotation about a particular axis. We adopt the convention that the $x_i$ system undergoes an infinitesimal rotation $\delta\vec{\theta}$ such that $\wrap{d\vec{r}}_{\text{fixed}} = d\vec{\theta}\times\vec{r}$. Differentiating gives $\wrap{d\vec{r}/dt}_{\text{fixed}} = \vec{\omega}\times\vec{r}$, for $P$ fixed in the $x_i$ system. Including motion in the rotating frame gives the general Eq. \ref{eq:cm-noninert-gen-vec}. Note that $\dot{\vec{\omega}}$ is the same in both frames, since $\vec{\omega}\times \vec{\omega} = 0$.
\texfig[0.4]{cm-noninert-setup}{The general configuration we use, where the $x_i$ system is rotating and the $x_i'$ is fixed in the ambient space. The rotating frame is related to the fixed frame by $\vec{R}$, such that $\vec{r}' = \vec{R} + \vec{r}$.}
\eqn[eq:cm-noninert-gen-vec]{\wrap{\frac{d\vec{Q}}{dt}}_{\text{fixed}} = \wrap{\frac{d\vec{Q}}{dt}}_{\text{rotating}} + \vec{\omega}\times\vec{Q}}
\eqn[eq:cm-noninert-vf]{\vec{v}_f = \vec{V} + \vec{v}_r + \vec{\omega}\times\vec{r}}
The velocity in the fixed frame $\vec{v}_f$ can be thus written in terms of the linear velocity of the moving origin $\vec{V}$, the velocity in the rotating frame $\vec{v}_r$ and the angular velocity of the rotating frame $\vec{\omega}$. The term $\vec{\omega}\times\vec{r}$ represents the velocity due to rotation of the moving axes.
% Centrifugal and Coriolis Forces
\subsection{Centrifugal and Coriolis Forces}
Deriving the expression for acceleration introduces a couple new terms, which end up becoming the Centrifugal and Coriolis forces when scaled by inertial mass.  Using the above, we can take time derivatives in the fixed frame to find $\dot{\vec{v}}_f = \dot{\vec{V}} + \dot{\vec{v}}_r + \dot{\vec{\omega}}\times\vec{r} + \vec{\omega}\times\dot{\vec{r}}$. These terms can be rearranged using the above relations to $\vec{F}_{\text{eff}} = \vec{F}_f - m\ddot{\vec{R}}_f - m\dot{\vec{\omega}}\times\vec{r} - m\vec{\omega}\times\wrap{\vec{\omega}\times\vec{r}} - 2m\vec{\omega}\times\vec{v}_r$, which gives the effective force perceived in the rotating frame. The latter two terms give the Centrifugal and Coriolis forces:
\eqn{\vec{F}_{\text{Cfl}} = -m\vec{\omega}\times \wrap{\vec{\omega}\times\vec{r}}, \quad\quad \vec{F}_{\text{Cor}} = -2m\vec{\omega}\times\vec{v}_r}
% Motion Relative to Earth
\subsection{Motion Relative to Earth}
When applied to a rotating frame on the Earth's surface, the above the centrifugal force term modifies the gravitational vector as $\vec{g} = \vec{g}_0 - \vec{\omega}\times\wraph{\vec{\omega}\times\wrap{\vec{r}+\vec{R}}}$, which defines "flat" surfaces. The Coriolis force causes a drift to right/left of velocity vector in northern/southern hemisphere.
\eqn{\vec{F}_{\text{eff}} = \vec{S} + m\vec{g} - 2m\vec{\omega}\times\vec{v}_r}



%%%%%%%%%%%%%%%%%%%%%%%%%%%%%%%%%%%%%%%%%
%             RIGID BODIES              %
%%%%%%%%%%%%%%%%%%%%%%%%%%%%%%%%%%%%%%%%%
\newpage
\section{Rigid Bodies}
\label{sec:cm-rigid}
Rigid bodies are physical objects with no relative internal motion, and can be modeled as either a system of particles whose relative distances are absolutely fixed, or as a matter-density field whose relative gradient is fixed. The motion of the body can be decomposed into that of its center of mass (CM) and rotation about that point. The position of the body can be described with 6 quantities, the $\RThree$ location of the CM, and the three Eulerian angles (defined below). Rotation in 3-dims requires a more complicated mathematical object, the \textit{inertia tensor}, which we define below.
% Inertia Tensor
\subsection{Inertia Tensor}
% Derivation
For a system of masses $m_{\alpha}$, we can use results from rotational reference frames to describe the velocity of the particles in the fixed frame $\vec{v}_{\alpha}$ in terms of the velocity of the CM $\vec{V}$ and the frame rotation $\omega$. Since the particles do not move in the body frame, the velocity takes a restricted form of Eq. \ref{eq:cm-noninert-vf}. Computing the kinetic energy, we observe that the \textit{components of the inertia tensor relate the rotational kinetic energy to the angular velocity}.
\eqn{\vec{v}_{\alpha} =\vec{V} + \vec{\omega}\times \vec{r}_{\alpha} \where\vec{v}_r = \wrap{\ddt[\vec{r}]}_{\text{rot}}\equiv 0}
\eqn[eq:cm-inert-tens-1]{T = \sum\limits_{\alpha} m_{\alpha}\vec{v}_{\alpha}^2 = \half\sum\limits_{\alpha}m_{\alpha}V^2 + \half\sum\limits_{\alpha}m_{\alpha}\wrap{\vec{\omega}\times\vec{r}_{\alpha}} = T_{\text{trans}} + T_{\text{rot}},\ \text{resp.}}
\eqn{\begin{split}
T_\text{rot} &= \half\sum\limits_{\alpha}m_{\alpha}\wraph{\omega^2r_{\alpha}^2 - \wrap{\vec{\omega}\cdot \vec{r}_{\alpha}}^2} = 
	\half\sum\limits_{\alpha}\sum\limits_{i,j}m_{\alpha}\wraph{\omega_i\omega_j\delta_{ij}\wrap{\sum\limits_k x_{\alpha, k}^2} - \omega_i\omega_j x_{\alpha,i}x_{\alpha,j}}\\
	&=\half\sum\limits_{i,j}\omega_i\omega_j\sum\limits_{\alpha}m_{\alpha}\wrap{\delta_{ij}\sum\limits_k x_{\alpha,k}^2 - x_{\alpha,i}x_{\alpha,j}} = \half\sum\limits_{\alpha}\tens{I}_{ij}\omega_i\omega_j
\end{split}}
\eqn{\boxed{
	\mathcal{I}_{ij}\equiv\sum\limits_{\alpha}m_{\alpha}\wrap{\delta_{ij}\sum\limits_k x_{\alpha,k}^2 - x_{\alpha,i}x_{\alpha,j}} = 
	\int_V\rho(\vec{r})\wrap{\delta_{ij}\sum_k x_k^2 - x_ix_j}dv
}}
Several notes on the above derivation. In Eq \ref{eq:cm-inert-tens-1} we are able to omit the cross-term $\sum_{\alpha}m_{\alpha}\vec{V}\cdot \vec{\omega}\times\vec{r}_{\alpha}$ because it can be factored to $\vec{V}\cdot\vec{\omega}\times\wrap{\sum_{\alpha}m_{\alpha}\vec{r}_{\alpha}} = \vec{V}\cdot\vec{\omega}\times M\vec{R}$, and $\vec{R} = 0$ since the origin of the rotating frame is the CM. In the following equation, $\vec{r}_{\alpha} = \wrap{x_{\alpha,1},x_{\alpha,2},x_{\alpha,3}}$, and we exploit the fact that $\wrap{\vec{A}\times\vec{B}}^2 = A^2B^2 - \wrap{\vec{A}\cdot \vec{B}}^2$, as well as $\omega_i = \sum_j \delta_{ij} \omega_j$. Note that in the most restricted (symmetric) case we recover $T_{\text{rot}} = \half I \omega^2$, where $I$ is the moment of inertia $I = \tens{I}_{11} = \tens{I}_{22} = \tens{I}_{33}$.

% Formal Definition
More formally, the inertia tensor $\tens{I} \in S^2\wrap{\RThree} \subset \mathcal{T}^{2}_{0}$ is a rank-2 symmetric tensor over the vector field $\RThree$, $\vec{I}: \RThree \times \RThree \rightarrow \R$, such that when acting on angular velocity vectors $\omega$, produces the rotational kinetic energy (doubled, since summing over all permutations of symmetric indices), $\tens{I}\wrap{\omega, \omega} = 2T_{\text{rot}}$. We can raise an index on $\tens{I}$ to produce a linear operator $\tens{I}: \RThree \to \RThree$ such that $\vec{L} = \tens{I}\vec{\omega}$. Using the metric dual map $L: \RThree \to \dual{\RThree}$ where $L(v) \equiv (v|\cdot)$ such that $L(v)(w)\equiv(v|w)$, the inertia tensor can be written:
\eqn{\boxed{\tens{I} = \sum\limits_{\alpha}m_{\alpha} \wrap{r_{\alpha}^2g - L\wrap{r_{\alpha}} \otimes L\wrap{r_{\alpha}}},\quad T_{\text{rot}}=\tens{I}_{ij}\omega^i\omega^j,\quad L^i = \tens{I}^{i}{}_{j}\omega^j = g^{il}\tens{I}_{lj}\omega^j}}
%   Similarity Transformation

\subsection{Principal Axes of Inertia}
%   Principal Inertia axes
The \textit{principal axes of inertia} are a basis $\{\hat{e}_i\}$ that diagonalize the inertia tensor, such that $\tens{I}_{ij}=\delta_{ij}I_i$, where the $i$th diagonal element is the $i$th moment of inertia. In this basis, the expressions for rotational kinetic energy and angular momentum are considerably simplified:
\eqn{T_\text{rot} = \half\sum_iI_i\omega_i^2,\quad\quad L_i = I_i\omega_i}
Determining the principal axes of inertia amounts to solving the characteristic equation (eigenvalue problem) of $\norm{\wraph{\tens{I}}_{\{\hat{e}_i\}} - \lambda} = 0$. The three eigenvalues are the \textit{principal moments of inertia} and correspond to the principal axes of inertia. If the rotation takes place about one of the principal axes, then the moment of inertia is completely specified by the corresponding principal moment of inertia, such that $\vec{L} = I_i\vec{\omega} \iff \hat{e}_{\text{rot}} = \hat{e}_i$. It is often possible to determine the principal axes by examining the symmetry of the body in question. Some special cases have names: $I_1 = I_2 = I_3$ is called \textit{spherical top}, $I_1 = I_2 \neq I_3$ is called \textit{symmetric top}, and $I_1 \neq I_2 \neq I_3$ (for any index permutation) is called \textit{asymmetric top}.
%   Diagonalization
For specific techniques on diagonalization, see App \ref{sec:app-lin-alg-legendre}.
%   Change of internal coordinates

\subsection{Eulerian Angles}
% Eulerian Angles
% Euler Equations for Rigid Body

\subsection{Examples}
% Symmetric Top Motion
% Stability of Rigid Body Rotation



%%%%%%%%%%%%%%%%%%%%%%%%%%%%%%%%%%%%%%%%%
%         COUPLED OSCILLATIONS          %
%%%%%%%%%%%%%%%%%%%%%%%%%%%%%%%%%%%%%%%%%
\newpage
\section{Coupled Oscillations}
\label{sec:cm-coupled}

 
\eqn{x}


%%%%%%%%%%%%%%%%%%%%%%%%%%%%%%%%%%%%%%%%%
%          CONTINUOUS SYSTEMS           %
%%%%%%%%%%%%%%%%%%%%%%%%%%%%%%%%%%%%%%%%%
\newpage
\section{Continuous Systems}
\label{sec:cm-continuous}

% Wave Equation
% Group velocity and dispersion relations


\eqn{x}

