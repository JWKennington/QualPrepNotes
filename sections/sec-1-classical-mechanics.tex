\newpage
%%%%%%%%%%%%%%%%%%%%%%%%%%%%%%%%%%%%%%%%%
%      CLASSICAL MECHANICS SECTION      %
%%%%%%%%%%%%%%%%%%%%%%%%%%%%%%%%%%%%%%%%%
\section{Classical Mechanics}
\label{sec:classmech}
The classical mechanics subject matter in the qualifying exam is primarily at the level of \cite{thorntonClassicalDynamicsParticles2004}, a common, upper-level undergraduate mechanics text.


%%%%%%%%%%%%%%%%%%%%%%%%%%%%%%%%%%%%%%%%%
%          NEWTONIAN MECHANICS          %
%%%%%%%%%%%%%%%%%%%%%%%%%%%%%%%%%%%%%%%%%
\subsection{Newtonian Mechanics}
Newtonian mechanics is typically covered in the introductory-level mechanics courses, but the basics are worth revisiting for completeness.

Objects do not change their motion unless acted on by external forces, and when acted upon, the acceleration is linearly proportional to the net applied force. The coefficient is \textbf{inertial mass}, not to be confused with other notions of "mass" as will be seen in Sec \ref{secSpecRel}

\eqn{\sum \vec{F} = \vec{F}_{net} = m\vec{a}}

The solution for Newtonian equations of motion under constant acceleration, for a general coordinate $q$ is given by: 

\eqn{q(t) = q_0 + \dot{q}t + \frac{1}{2}\ddot{q}t^2}


%%%%%%%%%%%%%%%%%%%%%%%%%%%%%%%%%%%%%%%%%
%          LAGRANGIAN MECHANICS         %
%%%%%%%%%%%%%%%%%%%%%%%%%%%%%%%%%%%%%%%%%
\subsection{Lagrangian Mechanics}


